%--------------------------------------%
%--------------------------------------%
%--------------------------------------%
%----- Pacotes e Classe do Documento --%
\RequirePackage[brazil]{babel}
\usepackage[T1]{fontenc}
\usepackage{ae}
\RequirePackage[latin1,utf8]{inputenc}
\RequirePackage[colorlinks=true]{hyperref}
\RequirePackage{amsmath,amsfonts,amsbsy,graphicx,tabularx,enumerate,fancyhdr}
\usepackage[dvipsnames,svgnames]{xcolor}
\usepackage{multicol,wrapfig}
\RequirePackage{amsthm,lipsum}
\RequirePackage{tikz}
\usetikzlibrary{shadows, shapes}
\usepackage{titlesec}
\RequirePackage{fourier-orns}
\usepackage{caption}
\usepackage{xcolor}
\usepackage{amsthm}
\usepackage{tikz}
\captionsetup{
  justification=raggedright,
  labelfont={color=Maroon,bf},
  font=footnotesize}
\fancyhead{}
\fancyfoot{}
\reversemarginpar
\pagestyle{fancy}
\titlespacing*{\chapter}{-120pt}{-50pt}{20pt}
\titlespacing*{\section}{0pt}{-20pt}{-20pt}
\titlespacing*{\subsection}{0pt}{0pt}{-5pt}
\titleformat{\chapter}[display]{\normalfont\huge\bfseries}{\filright\MakeUppercase{\chaptertitlename} \Huge\Roman{chapter}}{20pt}{\Huge}
%---------------------------------------------------------%
%---------------------------------------------------------%
%-------------------------------------%
%---------------------------------------------------------%
%---------------------------------------------------------%
%---------------------------------------------------------%
%---------------------------------------------------------%
%---------------------------------------------------------%
%-------------------------------------%
%-------ARQUIVO DE DEFINIÇÕES---------%
%----------- 07/10/08 ----------------%
%-------- Versão Original ------------%
%---- Jonas Renan Moreira Gomes ------%
%-------------------------------------%
\addtolength{\parskip}{0.5\baselineskip} %pula linha depois de parágrafo
%-------------------------------------%
%-----------SEPARADORES---------------%
%-------------------------------------%
%-------------------------------------%
%------------CONJUNTOS----------------%
%-------------------------------------%
\newcommand{\R}{\ensuremath{\mathbb{R}}}
\newcommand{\Z}{\ensuremath{\mathbb{Z}}}
\newcommand{\C}{\ensuremath{\mathbb{C}}}
\newcommand{\Rep}{\ensuremath{\mathbb{R}_+}}
\newcommand{\Na}{\ensuremath{\mathbb{N}}}
\newcommand{\Q}{\ensuremath{\mathbb{Q}}}

%-------------------------------------%
%-------- FRASES MAIS USADAS ---------%
%-------------------------------------%
\newcommand{\Fl}{\ensuremath{\Rightarrow}}
\newcommand{\talque}{\ensuremath{\text{ tal que }}}
\newcommand{\sol}[1]{\\ \textbf{Solução} #1}
\newcommand{\dica}{\\ \textit{Dica}\ }
\newcommand{\desafio}{\begin{exer}[Desafio]}
%-------------------------------------%
%-------------- SEÇÕES ---------------%
%-------------------------------------%
\theoremstyle{plain}
\newtheorem{teorema}{Teorema}
\newtheorem{afirmacao}{Afirmação}
\newtheorem{lema}{Lema}
\newtheorem{proposicao}{Proposição}
\newtheorem{corolario}{Corolário}
\newtheorem{propriedade}{Propriedade}
\newtheorem{inequacao}{Inequação}
\newtheorem{notacao}{Notação}
\theoremstyle{definition}
\newtheorem{exeresol}{Exercício Resolvido}
\newtheorem{exer}{\bfseries Exercício}
\newtheorem{inlinexer}{\color{BurntOrange}Testando o Conceito}
\newtheorem{exerfix}{Exercício de Fixação}
\newtheorem{dfn}{Definição}
\newtheorem{exemplo}{Exemplo}
%------------ Escrever na margem ------------------------%
\newcommand{\margem}[1]{\marginpar{
%desenha o retangulo
\begin{tikzpicture}
\node[top color=LightBlue, bottom color = LightBlue, thick,rectangle]{};
\end{tikzpicture}
%terminou o retangulo
\footnotesize #1
}
}
%============= Página de exercícios =====================%
\newcommand{\comecaexer}{
\newgeometry{inner=2cm,outer=2cm,bottom=2cm,top=2cm}
\begin{multicols}{2}\noindent
}
\newcommand{\terminaexer}{
\end{multicols}\noindent
\restoregeometry
}
%--------------------------------------------------------%
%-------------ARQUIVOS DE MODELOS PARA AS----------------%
%----------------CAIXAS FLUTUANTES-----------------------%
%--------------------------------------------------------%
%----------------CAIXA TESTE ----------------------------%
%--------------------------------------------------------%
%--------------------------------------------------------%
\newcommand{\propriedades}[2]{
%---------- \propriedades{nome longo}{propriedade em si}-----------------%
\tikzstyle{mybox} = [draw=black,top color=BurntOrange!5,
  bottom color=BurntOrange!0,rounded corners,thick,rectangle,inner sep=10pt, inner ysep=20pt]
\tikzstyle{fancytitle} =[fill=BurntOrange!5, text=black, draw=black]
\begin{tikzpicture}
\node [mybox] (box){
    \begin{minipage}{0.80\textwidth}
       #2
    \end{minipage}
};
\node[fancytitle, right=10pt,thick, inner sep = 8pt] at (box.north west){#1};

\end{tikzpicture}%
}
%--------------------------------------------------------%
%--------------------------------------------------------%
%--------------------------------------------------------%

%------------- Caixa de Propriedades --------------------%
\newcommand{\caixaprop}[2][0.9\textwidth]{
 \par\noindent\tikzstyle{mybox} = [draw=black,top color=BurntOrange!5,
  bottom color=BurntOrange!0,rounded corners,thick,rectangle,inner sep=10pt]
 \begin{tikzpicture}
  \node [mybox] (box){%
   \begin{minipage}{#1}{#2}\end{minipage}
  };
 \end{tikzpicture}
}
%------------- Caixa da Definição --------------------%
\newcommand{\caixa}[2][0.9\textwidth]{
 \par\noindent\tikzstyle{mybox} = [top color=gray!5,
  bottom color=gray!5,thick,rectangle,inner sep=10pt]
 \begin{tikzpicture}
  \node [mybox] (box){%
   \begin{minipage}{#1}{#2}\end{minipage}
  };
 \end{tikzpicture}
}

%------------- Caixa da Sessão --------------------%
\newcommand{\caixasessao}[2][\textwidth]{
 \par\noindent\tikzstyle{mybox} = [draw=black,left color=white,
  right color=white,rounded corners,thick,rectangle,inner sep=5pt]
 \begin{tikzpicture}
  \node [mybox, drop shadow] (box){%
   \begin{minipage}{#1}{#2}\end{minipage}
  };
 \end{tikzpicture}
}

%------------- Caixa Margem --------------------%
\definecolor{amarelinho}{RGB}{237,242,64}
\newcommand{\caixamargem}[2][\marginparwidth]{
 \par\noindent
 \tikzstyle{mybox} = [draw=white, top color=amarelinho, bottom color=amarelinho,thick,rectangle,inner sep=15pt]
 \begin{tikzpicture}
  \node [mybox] (box){
   \begin{minipage}{#1}{#2 \newline \newline \newline}
   \end{minipage}
  };
  \node[above left, outer sep=-3pt] at (box.south east) { \includegraphics{postit.png} };
  \node[outer sep=-3pt] at (box.north) { \includegraphics{tachinha.png} };
 \end{tikzpicture}
}

%---------- Caixa de Exemplo -----------------
%---------------------------------------------
\newcounter{NumeroExemplo}[chapter]
\newcommand{\exem}[2]{\stepcounter{NumeroExemplo}


\begin{tikzpicture}
\node[top color=LightBlue, bottom color = LightBlue, rounded corners, thick,rectangle] (caixa) {\textbf{Exemplo \arabic{NumeroExemplo}}};
\node[rectangle, right] at (caixa.east) {\textbf{#1}};
\end{tikzpicture}\newline
\normalsize #2

\begin{center}
\begin{tikzpicture}
\draw[fill=LightBlue,drop shadow] (0,0) circle [radius=0.1];
\draw[fill=LightBlue,drop shadow] (3,0) circle [radius=0.1];
\draw[fill=LightBlue,drop shadow] (6,0) circle [radius=0.1];
\draw[fill=LightBlue,drop shadow] (9,0) circle [radius=0.1];
\end{tikzpicture}
\end{center}
}
%--------------------------------------
%------- Caixa de Definição -----------
\newcommand{\definicao}[1]{
	\caixa
	{
		\begin{dfn}
				#1
		\end{dfn}
	}
}
\newcommand{\frapar}[2]{ \ensuremath{\left ( \frac{#1}{#2} \right)}}
\titleformat{\section}
{\Large\bfseries}%O que vem antes do texto <format>
{}%{Label}
{0em}%Separção
{\caixasessao}%Código antes!!
\titleformat{\subsection}
{\color{MidnightBlue}\large\normalfont\bfseries}
{\color{MidnightBlue}}{1 pt}{}
\renewcommand{\sectionmark}[1]{\markright{#1}}
%--------------------------------------------------------
%------------------------------------------------------
%--- MONTA O DOC ---------------------------------------
