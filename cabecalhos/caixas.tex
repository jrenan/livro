%--------------------------------------------------------%
%-------------ARQUIVOS DE MODELOS PARA AS----------------%
%----------------CAIXAS FLUTUANTES-----------------------%
%--------------------------------------------------------%
%----------------CAIXA TESTE ----------------------------%
%--------------------------------------------------------%
%--------------------------------------------------------%
\newcommand{\propriedades}[2]{
%---------- \propriedades{nome longo}{propriedade em si}-----------------%
\tikzstyle{mybox} = [draw=black,top color=BurntOrange!5,
  bottom color=BurntOrange!0,rounded corners,thick,rectangle,inner sep=10pt, inner ysep=20pt]
\tikzstyle{fancytitle} =[fill=BurntOrange!5, text=black, draw=black]
\begin{tikzpicture}
\node [mybox] (box){
    \begin{minipage}{0.80\textwidth}
       #2
    \end{minipage}
};
\node[fancytitle, right=10pt,thick, inner sep = 8pt] at (box.north west){#1};

\end{tikzpicture}%
}
%--------------------------------------------------------%
%--------------------------------------------------------%
%--------------------------------------------------------%

%------------- Caixa de Propriedades --------------------%
\newcommand{\caixaprop}[2][0.9\textwidth]{
 \par\noindent\tikzstyle{mybox} = [draw=black,top color=BurntOrange!5,
  bottom color=BurntOrange!0,rounded corners,thick,rectangle,inner sep=10pt]
 \begin{tikzpicture}
  \node [mybox] (box){%
   \begin{minipage}{#1}{#2}\end{minipage}
  };
 \end{tikzpicture}
}
%------------- Caixa da Definição --------------------%
\newcommand{\caixa}[2][0.9\textwidth]{
 \par\noindent\tikzstyle{mybox} = [top color=gray!5,
  bottom color=gray!5,thick,rectangle,inner sep=10pt]
 \begin{tikzpicture}
  \node [mybox] (box){%
   \begin{minipage}{#1}{#2}\end{minipage}
  };
 \end{tikzpicture}
}

%------------- Caixa da Sessão --------------------%
\newcommand{\caixasessao}[2][\textwidth]{
 \par\noindent\tikzstyle{mybox} = [draw=black,left color=white,
  right color=white,rounded corners,thick,rectangle,inner sep=5pt]
 \begin{tikzpicture}
  \node [mybox, drop shadow] (box){%
   \begin{minipage}{#1}{#2}\end{minipage}
  };
 \end{tikzpicture}
}

%------------- Caixa Margem --------------------%
\definecolor{amarelinho}{RGB}{237,242,64}
\newcommand{\caixamargem}[2][\marginparwidth]{
 \par\noindent
 \tikzstyle{mybox} = [draw=white, top color=amarelinho, bottom color=amarelinho,thick,rectangle,inner sep=15pt]
 \begin{tikzpicture}
  \node [mybox] (box){
   \begin{minipage}{#1}{#2 \newline \newline \newline}
   \end{minipage}
  };
  \node[above left, outer sep=-3pt] at (box.south east) { \includegraphics{postit.png} };
  \node[outer sep=-3pt] at (box.north) { \includegraphics{tachinha.png} };
 \end{tikzpicture}
}

%---------- Caixa de Exemplo -----------------
%---------------------------------------------
\newcounter{NumeroExemplo}[chapter]
\newcommand{\exem}[2]{\stepcounter{NumeroExemplo}


\begin{tikzpicture}
\node[top color=LightBlue, bottom color = LightBlue, rounded corners, thick,rectangle] (caixa) {\textbf{Exemplo \arabic{NumeroExemplo}}};
\node[rectangle, right] at (caixa.east) {\textbf{#1}};
\end{tikzpicture}\newline
\normalsize #2

\begin{center}
\begin{tikzpicture}
\draw[fill=LightBlue,drop shadow] (0,0) circle [radius=0.1];
\draw[fill=LightBlue,drop shadow] (3,0) circle [radius=0.1];
\draw[fill=LightBlue,drop shadow] (6,0) circle [radius=0.1];
\draw[fill=LightBlue,drop shadow] (9,0) circle [radius=0.1];
\end{tikzpicture}
\end{center}
}
%--------------------------------------
%------- Caixa de Definição -----------
\newcommand{\definicao}[1]{
	\caixa
	{
		\begin{dfn}
				#1
		\end{dfn}
	}
}