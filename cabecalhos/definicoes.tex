%-------------------------------------%
%-------ARQUIVO DE DEFINIÇÕES---------%
%----------- 07/10/08 ----------------%
%-------- Versão Original ------------%
%---- Jonas Renan Moreira Gomes ------%
%-------------------------------------%
\addtolength{\parskip}{0.5\baselineskip} %pula linha depois de parágrafo
%-------------------------------------%
%-----------SEPARADORES---------------%
%-------------------------------------%
%-------------------------------------%
%------------CONJUNTOS----------------%
%-------------------------------------%
\newcommand{\R}{\ensuremath{\mathbb{R}}}
\newcommand{\C}{\ensuremath{\mathbb{C}}}
\newcommand{\Rep}{\ensuremath{\mathbb{R}^+}}
\newcommand{\Na}{\ensuremath{\mathbb{N}}}
\newcommand{\Q}{\ensuremath{\mathbb{Q}}}

%-------------------------------------%
%-------- FRASES MAIS USADAS ---------%
%-------------------------------------%
\newcommand{\paratodoepsilon}{\ensuremath{\forall \epsilon \in \Rep}}
\newcommand{\existedelta}{\ensuremath{\exists \delta \in \Rep}}
\newcommand{\existenzero}{\ensuremath{\exists n_0 \in \mathbb{N} \talque n > n_0}}
\newcommand{\talque}{\ensuremath{\text{ tal que }}}
\newcommand{\sol}{\\ \textbf{Solução}\ }
\newcommand{\dica}{\\ \textit{Dica}\ }
\newcommand{\desafio}{\begin{exer} \textbf{(Desafio)}}
%-------------------------------------%
%-------------- SEÇÕES ---------------%
%-------------------------------------%
\theoremstyle{plain}
\newtheorem{exeresol}{Exercício Resolvido}
\newtheorem{exer}{Exercício}
\newtheorem{exerfix}{Exercício de Fixação}
\newtheorem{teorema}{Teorema}
\newtheorem{afirmacao}{Afirmação}
\newtheorem{lema}{Lema}
\newtheorem{proposicao}{Proposição}
\newtheorem{corolario}{Corolário}
\newtheorem{propriedade}{Propriedade}
\newtheorem{inequacao}{Inequação}
\theoremstyle{definition}
\newtheorem{definicao}{Definição}
\newtheorem{exemplo}{Exemplo}